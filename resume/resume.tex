%%%% Compile with: texi2pdf resume.tex ## requires texlive (and maybe texinfo)

\documentclass[margintitle,line]{res}
\usepackage[a4paper,left=1.0cm,right=4.5cm,top=1.5cm,bottom=1.5cm]{geometry}
\usepackage[utf8]{inputenc}
\usepackage[colorlinks=true,urlcolor=blue]{hyperref}
%\usepackage{ifthen}
%\usepackage[pdftex]{graphicx}
%\usepackage{wrapfig}

%\newboolean{@all-talks}
%\setboolean{@all-talks}{false}

% specify some fonts and colors
\renewcommand{\familydefault}{\sfdefault}
\renewcommand{\sectionfont}{\scshape}
\renewcommand{\namefont}{\LARGE \bfseries}
\renewcommand{\titlefont}{\bf}
\renewcommand{\datesfont}{\bf}
\definecolor{linecolor}{RGB}{25,25,112}

% override defaults
\renewcommand{\employerfont}{}

% ugly kludge: we really have to define proper subsections in the .cls file
\renewcommand{\subsection}[1]{\section{\normalfont #1}}

\setlength{\parskip}{2ex}

\hypersetup{
  pdfauthor   = {Brit Butler},
  pdftitle    = {Curriculum Vitae: Brit Butler},
  pdfsubject  = {Curriculum Vitae},
  pdfkeywords = {Brit Butler, Curriculum Vitae, CV, resume},
}

\begin{document}

\name{Brit Butler}

\begin{resume}

% Specify the format of work entries
\begin{format}
\dates{l}\\
\title{l}\employer{r}\\
\body\\
\end{format}

%%%%%%%%%%%%%%%%%%%%%%%%%%%%%%%%%%%%%%%%%%%%%%%%%%%%%%%%%%%%%%%%%%%%%%%%%%%

%\begin{wrapfigure}{r}{40mm}
% \begin{center}
%  \includegraphics[width=0.1\textwidth]{lks}
% \end{center}
%\end{wrapfigure}

\section{Contact Information}

404-718-9378 \hfill \href{mailto:brit@kingcons.io}{\nolinkurl{brit@kingcons.io}} \\
866 Woodland Ave SE \hfill \url{https://github.com/kingcons/} \\
Atlanta, GA 30316 \hfill \url{https://blog.kingcons.io/} \\

%%%%%%%%%%%%%%%%%%%%%%%%%%%%%%%%%%%%%%%%%%%%%%%%%%%%%%%%%%%%%%%%%%%%%%%%%%%

\section{Work Experience}

\title{Engineering Manager}
\employer{\llap{Calendly}}
\dates{January 2022 -- May 2024}
\begin{position}
  I was the EM for 3 squads totaling a dozen direct reports, hiring half of those engineers to scale the squads.
  I oversaw the delivery of work such as our Calendly Analytics product, Self-Service Data Deletion, Activity Log, Calendar Service extraction, and Identity Service extraction.
  I consistently advocated for addressing tech debt whether through reviving the App Design guild, talking with product about sunsetting our Teams feature, or driving work to simplify the ownership model around our core data type.
  I improved internal communication to empower engineers, from a living document in Confluence for tracking Feature Ownership, to recurring weekly meetings with myself and Staff Engineers to unblock squads on Service Extraction, to running an annual Advent of Code channel encouraging engineers to have fun solving problems and trying new design patterns.
\end{position}

\title{Senior Software Engineer}
\employer{\llap{Calendly}}
\dates{September 2019 -- December 2021}
\begin{position}
  I wrote implementation plans and guided development for features such as the Share Modal, One-off Meetings, Double Booking Rules, and Reserved Times.
  I founded the Flamingo squad and acted as Team lead, running all agile ceremonies, coordinating with the PM around estimates and roadmap decisions, and coordinating the division of work on the squad.
  I participated in over 100 interviews and hiring panels as the org scaled.
  I served as an onboarding buddy to multiple junior and mid-level engineers.
  I organized and ran a Book Club to learn Rust that led to a half dozen engineers gaining proficiency.
\end{position}

\title{Lead Instructor}
\employer{\llap{Flatiron School}}
\dates{October 2018 -- Present}
\begin{position}
  I taught 80+ full-time students Ruby/Rails and React while managing and mentoring TAs.
  I developed bonus lectures and additional curriculum on unit testing, test coverage, and
  refactoring. I participated in internal discussions to improve our assessments and
  developed custom plans for struggling students.
\end{position}

\title{Senior Software Engineer}
\employer{\llap{Showcase IDX}}
\dates{June 2017 -- October 2018}
\begin{position}
  I took over Backend duties and DevOps for a Real Estate Search startup powering over 1000
  websites and helped scale the system to 1M+ page views a day. Major initiatives included moving
  the postgres instance from kubernetes to a dedicated Cloud SQL instance, writing a migration
  tool for customers on the legacy version of the product, improving the ingestion of data from
  MLS listing feeds, rebuilding the Elasticsearch cluster after a canary deploy went wrong,
  optimizing queries on our 30M+ record leads table, and other assorted production work.
  I did some ``Programmer Archaeology'' to relearn how to deploy the legacy system
  and switched it to using LetsEncrypt for SSL. Since I was the sole ops engineer, I thoroughly
  documented the architecture, its performance characteristics, and some gotchas before leaving.
\end{position}

\title{Lead Instructor}
\employer{\llap{The Iron Yard}}
\dates{December 2014 -- May 2017}
\begin{position}
  I taught immersive, full-time courses in Backend Engineering using Ruby/Rails and Frontend Engineering using Javascript/Angular.
  I was promoted to Lead Instructor after 6 months. As an instructor, I iterated
  on curriculum, lectured, graded assignments, and worked 1-on-1 with students
  during lab time. As a lead, I mentored a dozen new instructors at different
  campuses and served as a resource for classroom issues, struggling students,
  and other problems.
\end{position}

\title{Software Engineer}
\employer{\llap{Emcien}}
\dates{May 2013 -- October 2014}
\begin{position}
  I worked on several data analysis products written in Ruby/Rails and C.
  I added allocation tracking machinery to a modern 20k SLOC C project to aid
  in finding memory leaks and reducing the overall memory footprint. I took
  over maintenance of a legacy product, Mix, migrating from Ruby 1.8.7 to 1.9.3
  and overseeing numerous point releases. I also contributed substantial work
  to the primary product, Patterns, including the report download builders and
  storing report attributes in SQL shards.
\end{position}

%%%%%%%%%%%%%%%%%%%%%%%%%%%%%%%%%%%%%%%%%%%%%%%%%%%%%%%%%%%%%%%%%%%%%%%%%%%

\pagebreak

% listings
\setlength{\parskip}{1ex}

%%%%%%%%%%%%%%%%%%%%%%%%%%%%%%%%%%%%%%%%%%%%%%%%%%%%%%%%%%%%%%%%%%%%%%%%%%%

\subsection{Passion Projects}

\title{Author}
\employer{\href{https://kingcons.github.io/advent-of-code/}{advent-of-code}}
\dates{2021 -- 2023}
\begin{position}
  For the past several years, I have enjoyed participating in Advent of Code and
  organizing a group internal to my job to work through problems as well.

  On an group level, the purpose is not to compete but to remember the joy of solving
  tricky problems and encourage discussion among engineers about different approaches.

  On an individual level, I enjoy writing clean, readable solutions, with automated
  benchmarking and documentation generation shown in the site linked above.
\end{position}

\title{Author}
\employer{\href{https://github.com/kingcons/rawbones}{rawbones}}
\dates{May 2019 -- December 2019}
\begin{position}
  Rawbones is an NES emulator written in ReasonML and compiled to Javascript. It
  currently powers a React-based frontend (also in ReasonML) written in collaboration
  with my dear friend \href{https://jdabbs.com}{James Dabbs}. That frontend is called
  \href{https://github.com/jamesdabbs/epiderNES}{epiderNES} and can be seen in action
  \href{https://kingcons.io/epiderNES/}{here}.

  I later implemented another \href{https://github.com/kingcons/clones}{NES emulator in Common Lisp} incorporating some
  lessons from Rawbones around how synchronization between the CPU and PPU was
  performed. I streamed most of the implementation work on Twitch to about 60
  folks and have recordings that I need to upload.
\end{position}

\title{Author}
\employer{\href{https://github.com/kingcons/study-group}{study-group}}
\dates{July 2018 -- November 2018}
\begin{position}
  I have organized and led an \href{https://sarabander.github.io/sicp/}{SICP} study group for
  myself and former coworkers and students. In addition to providing code feedback,
  I prepare the schedule, a meeting location, and host weekly discussion both online and in person.
\end{position}

\title{Author}
\employer{\href{https://github.com/kingcons/salty-runbooks}{salty-runbooks}}
\dates{November 2014 -- September 2017}
\begin{position}
  A set of Ansible playbooks to better familiarize myself with configuration management
  and help automate the setup and administration of my personal servers. It includes roles
  for retrieving SSL certificates with LetsEncrypt, setting up Postfix as a mail server,
  running an IRC bouncer, a blog generated with Lisp and served by nginx, and various
  web applications written in PHP, Python, and Node for streaming music and hosting media.
\end{position}

\title{Author}
\employer{\href{https://github.com/kingcons/coleslaw}{coleslaw}}
\dates{August 2012 -- November 2014}
\begin{position}
  coleslaw is static blogware a la Jekyll, written in Common Lisp. It
  supports publishing via git push, markdown with code highlighting extensions,
  extensible content types, theming, and plugins for additional functionality.
  It is also a good example of Object Oriented Programming in Lisp. Since ``you are what you document'' there is a thorough  \href{https://github.com/kingcons/coleslaw/blob/master/docs/hacking.md}{Hacker's Guide to Coleslaw}
  explaining its internal design in addition to documentation of the Plugin API
  and extensions.
\end{position}

\title{Author}
\employer{\href{https://github.com/kingcons/cl-6502}{cl-6502}}
\dates{May 2011 -- May 2014}
\begin{position}
  cl-6502 is a MOS 6502 emulator, assembler, and disassembler written in
  Common Lisp. Inspired by Luke Gorrie's call for ``Readable Programs'' there
  is an \href{http://redlinernotes.com/docs/cl-6502.pdf}{annotated book} of the
  source code.  Some of the motivations behind cl-6502's creation are described
  \href{https://blog.kingcons.io/posts/Towards-Comprehensible-Computing.html}{here}.
  There is also a recorded talk on the project and related ideas called
  \href{https://vimeo.com/redline6561/on-programmer-archaeology}{``On Programmer Archaeology''}.
\end{position}

%%%%%%%%%%%%%%%%%%%%%%%%%%%%%%%%%%%%%%%%%%%%%%%%%%%%%%%%%%%%%%%%%%%%%%%%%%%

\section{Education}

\title{B.Sc. in Computer Science}
\employer{\llap{Southern Polytechnic State University}}
\dates{January 2009 -- May 2011}
\begin{position}
\end{position}

%%%%%%%%%%%%%%%%%%%%%%%%%%%%%%%%%%%%%%%%%%%%%%%%%%%%%%%%%%%%%%%%%%%%%%%%%%%

\section{Skills}

Programming Languages (intermediate): Common Lisp, Python, Ruby, Javascript \\
Programming Languages (novice): C, Scheme, Haskell, Factor, Elisp, Clojure \\
Markup Languages: HTML, CSS, SASS, LaTeX \\
Operating Systems: Mac OS X, Windows 3.11-7, Various Linux distributions
esp. Debian, Archlinux, Guix \\
Preferred Tools: Emacs, Git, Steel Bank Common Lisp, Debian \\

%%%%%%%%%%%%%%%%%%%%%%%%%%%%%%%%%%%%%%%%%%%%%%%%%%%%%%%%%%%%%%%%%%%%%%%%%%%

\section{Interests}

When I'm not in front of my computer, I like to make cocktails, spend
time with my partner and our goofy dogs, play Smash Brothers Melee,
and noodle on a modular synth. As far as Comp Sci topics, I am fascinated
by the implementation of dynamic, reflective languages such as Lisp and Smalltalk
and the careful interplay between compiler and runtime to make them fast.

%%%%%%%%%%%%%%%%%%%%%%%%%%%%%%%%%%%%%%%%%%%%%%%%%%%%%%%%%%%%%%%%%%%%%%%%%%%

\end{resume}
\end{document}
