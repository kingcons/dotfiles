%%%% Compile with: texi2pdf resume.tex ## requires texlive,

\documentclass[margintitle,line]{res}
\usepackage[a4paper,left=1.0cm,right=4.5cm,top=1.5cm,bottom=1.5cm]{geometry}
\usepackage[utf8]{inputenc}
\usepackage[colorlinks=true,urlcolor=blue]{hyperref}
%\usepackage{ifthen}
%\usepackage[pdftex]{graphicx}
%\usepackage{wrapfig}

%\newboolean{@all-talks}
%\setboolean{@all-talks}{false}

% specify some fonts and colors
\renewcommand{\familydefault}{\sfdefault}
\renewcommand{\sectionfont}{\scshape}
\renewcommand{\namefont}{\LARGE \bfseries}
\renewcommand{\titlefont}{\bf}
\renewcommand{\datesfont}{\bf}
\definecolor{linecolor}{RGB}{25,25,112}

% override defaults
\renewcommand{\employerfont}{}

% ugly kludge: we really have to define proper subsections in the .cls file
\renewcommand{\subsection}[1]{\section{\normalfont #1}}

\setlength{\parskip}{2ex}

\hypersetup{
  pdfauthor   = {Brit Butler},
  pdftitle    = {Curriculum Vitae: Brit Butler},
  pdfsubject  = {Curriculum Vitae},
  pdfkeywords = {Brit Butler, Butler, Curriculum Vitae, CV, resume},
}

\begin{document}

\name{Brit Butler}

\begin{resume}

% Specify the format of work entries
\begin{format}
\dates{l}\\
\title{l}\employer{r}\\
\body\\
\end{format}

%%%%%%%%%%%%%%%%%%%%%%%%%%%%%%%%%%%%%%%%%%%%%%%%%%%%%%%%%%%%%%%%%%%%%%%%%%%

%\begin{wrapfigure}{r}{40mm}
% \begin{center}
%  \includegraphics[width=0.1\textwidth]{lks}
% \end{center}
%\end{wrapfigure}

\section{Contact Information}

404-718-9378 \hfill {E-mail:} \href{mailto:redline6561@gmail.com}{\nolinkurl{redline6561@gmail.com}} \\
1270 West Peachtree Street, Apt 12G \hfill {Code:} \url{http://github.com/redline6561/} \\
Atlanta, GA 30309 \hfill {Web:} \url{http://www.redlinernotes.com/} \\

%%%%%%%%%%%%%%%%%%%%%%%%%%%%%%%%%%%%%%%%%%%%%%%%%%%%%%%%%%%%%%%%%%%%%%%%%%%

\section{\mbox{Work Experience}}

\subsection{Remunerated}

\title{Senior Software Engineer}
\employer{\llap{Primedia Inc.}}
\dates{September 2012 -- present}
\begin{position}
  I worked on a Riak-backed configuration service using Clojure and the Noir web framework. It was later
  refactored to remove Noir when the maintainers deprecated it. I also attended Strange Loop
  in St. Louis and took \href{http://blog.redlinernotes.com/posts/Strange-Loop-Notes---Day-2.html}{lots}
  of \href{http://blog.redlinernotes.com/posts/Strange-Loop-Notes---Day-1.html}{notes}.
\end{position}

\title{Software Engineer}
\employer{\llap{CMGdigital}}
\dates{May 2011 -- September 2012}
\begin{position}
  I worked on a 160k SLOC Django project to serve Newspaper, TV, and Radio publishers.
  I had particular focuses on admin customization and data migration. I also gave talks
  on \href{http://redlinernotes.com/docs/talks/opa.html}{``Programmer Archaeology''},
  \href{http://redlinernotes.com/docs/talks/wosw.html}{the Economics of Open Source}, and
  \href{http://redlinernotes.com/docs/talks/cl-gbu.html}{the Common Lisp programming language}.
\end{position}

\title{Web Developer (Part-time)}
\employer{\llap{Teacher Resource Network}}
\dates{August 2010 -- November 2010}
\begin{position}
  I worked closely with the Lead Developer writing extension modules
  for the Drupal Content Management System in PHP. I also worked on
  various UI aspects of the site using Javascript and jQuery.
  I left to focus my full attention on school and personal projects.
\end{position}

%%%%%%%%%%%%%%%%%%%%%%%%%%%%%%%%%%%%%%%%%%%%%%%%%%%%%%%%%%%%%%%%%%%%%%%%%%%

\subsection{Voluntary}

%% Also slideware, romreader, cl-echonest, etc...

\title{Lead Developer, Maintainer}
\employer{\href{http://github.com/redline6561/coleslaw}{coleslaw}}
\dates{August 2012 -- present}
\begin{position}
  coleslaw is static blogware a la Jekyll, written in Common Lisp. It supports
  publishing via git push, RSS+Atom feeds, markdown with code highlighting extensions,
  comments via disqus, analytics via google, and plugin and theming support. It presently
  powers my \href{http://blog.redlinernotes.com/}{personal blog}.
\end{position}

\title{Lead Developer, Maintainer}
\employer{\href{http://github.com/redline6561/cl-6502}{cl-6502}}
\dates{May 2011 -- present}
\begin{position}
  cl-6502 is a MOS 6502 emulator, assembler, and disassembler written in Common Lisp.
  It strives for concision, correctness, performance and extensibility. Some of the
  motivations behind its creation and reasoning behind its design are described
  \href{http://blog.redlinernotes.com/posts/On-Interactive-Retrocomputing.html}{here}.
  There is also a recorded talk on the project and related ideas called
  \href{http://vimeo.com/redline6561/on-programmer-archaeology}{``On Programmer Archaeology''}.
\end{position}

\title{Lead Developer, Maintainer}
\employer{\href{http://github.com/redline6561/cl-scrobbler}{cl-scrobbler}}
\dates{September 2011 -- October 2011}
\begin{position}
  cl-scrobbler is a Lisp library for interaction with last.fm's
  \href{http://www.last.fm/api/scrobbling}{Scrobbling API}. I have integrated
  it with the \href{http://vintage-digital.com/hefner/software/shuffletron/}{shuffletron}
  music player and use it daily. I also
  \href{http://redlinernotes.com/docs/talks/cl-mft.html}{delivered a talk}
  about how lisp helped me integrate and debug the two at a meeting of the
  Atlanta Lisp Users Group.
\end{position}

\title{Developer, Maintainer}
\employer{\href{http://github.com/redline6561/paktahn}{Paktahn}}
\dates{September 2009 -- present}
\begin{position}
  Paktahn is a command-line based package management helper for Archlinux.
  I have made 8 major releases, fixed numerous bugs, implemented features
  including proxy support and AUR updates and assisted users on the Archlinux
  forums. Recently, I handled the migration from version 5 to 6 of the
  underlying C library (libalpm) and found a regression in libalpm's API in
  process.
\end{position}

\title{Contributor}
\employer{Miscellaneous}
\begin{position}
  I have also contributed feature, portability, and documentation patches to:
  \begin{itemize}
    \item{SBCL, the Steel Bank Common Lisp compiler}
    \item{st-json, a Common Lisp JSON library}
    \item{weblocks, a Common Lisp Web Framework}
    \item{kardboard, a virtual Kanban web app made with Python and Flask}
    \item{pybrightcove, a Python interface to Brightcove's API}
    \item{external-program, a Common Lisp library for running external processes}
  \end{itemize}
\end{position}

%%%%%%%%%%%%%%%%%%%%%%%%%%%%%%%%%%%%%%%%%%%%%%%%%%%%%%%%%%%%%%%%%%%%%%%%%%%

\section{Education}

\title{B.Sc. in Computer Science}
\employer{\llap{Southern Polytechnic State University}}
\dates{January 2009 -- May 2011}
\begin{position}
  I transferred to SPSU in August 2007 to pursue Computer Science.
  After the first semester, I took a year off to work full-time and self-study.
  I returned in January 2009 and graduated in May 2011.
\end{position}

\title{Self-Study}
\dates{January 2008 -- August 2008}
\begin{position}
  After my first semester at SPSU I decided to self-study
  programming while working full-time. I did this
  throughout 2008 and have written about it
  (including posting my work progress), notably
  \href{http://blog.redlinernotes.com/posts/Leaving-College-to-Leverage-Compulsion.html}{here} and
  \href{http://blog.redlinernotes.com/posts/Spring-2008-Schedule-and-Syllabus.html}{here}.
  Example work from self-study can be found
  \href{http://blog.redlinernotes.com/posts/SICP-Section-13.html}{here}.
\end{position}

\title{B.A. in Literature (transferred)}
\employer{\llap{Oglethorpe University}}
\dates{August 2004 -- May 2007}
\begin{position}
  I spent several years at Oglethorpe studying literature
  and economics with particular interests in 20th century
  poetry and the provision of public goods before leaving
  because OU did not offer a CS program.
\end{position}

%%%%%%%%%%%%%%%%%%%%%%%%%%%%%%%%%%%%%%%%%%%%%%%%%%%%%%%%%%%%%%%%%%%%%%%%%%%

% listings
\setlength{\parskip}{1ex}

%%%%%%%%%%%%%%%%%%%%%%%%%%%%%%%%%%%%%%%%%%%%%%%%%%%%%%%%%%%%%%%%%%%%%%%%%%%

\section{Skills}

I've loved computers as long as I can remember. I've used Linux exclusively
since 2006 and began programming in 2008. I have a particular interest in
compilers and the implementation of dynamic, interactive languages but am
consistently seeking to extend my knowledge in all areas of Software Development
and Computer Science.

Programming Languages (intermediate): Common Lisp, Python \\
Programming Languages (novice): C, Java, PHP, Erlang, Scheme, C\#, Haskell,
C++, Factor, Javascript, Emacs Lisp, Clojure \\
Markup Languages: HTML, CSS, LaTeX \\
Operating Systems: Mac OS X, Windows 3.11-7, Various Linux distributions
esp. Debian and Archlinux \\
Preferred Tools: Emacs, Git, Steel Bank Common Lisp, Debian \\

%%%%%%%%%%%%%%%%%%%%%%%%%%%%%%%%%%%%%%%%%%%%%%%%%%%%%%%%%%%%%%%%%%%%%%%%%%%

\section{Interests}

When I'm not in front of my computer, I like to skateboard, make
\href{http://soundcloud.com/redlinernotes}{mixtapes}, play blues guitar, and
read and write poetry. I also try to regularly find the time to blog about
these thoughts and interests at \url{http://blog.redlinernotes.com/}.

I have serious interests in Intellectual Property Law and Peer Production.

%%%%%%%%%%%%%%%%%%%%%%%%%%%%%%%%%%%%%%%%%%%%%%%%%%%%%%%%%%%%%%%%%%%%%%%%%%%

\section{References}

Available on request.

\end{resume}
\end{document}
