%%%% Compile with: texi2pdf resume.tex ## requires texlive,

\documentclass[margintitle,line]{res}
\usepackage[a4paper,left=1.0cm,right=4.5cm,top=1.5cm,bottom=1.5cm]{geometry}
\usepackage[utf8]{inputenc}
\usepackage[colorlinks=true,urlcolor=blue]{hyperref}
%\usepackage{ifthen}
%\usepackage[pdftex]{graphicx}
%\usepackage{wrapfig}

%\newboolean{@all-talks}
%\setboolean{@all-talks}{false}

% specify some fonts and colors
\renewcommand{\familydefault}{\sfdefault}
\renewcommand{\sectionfont}{\scshape}
\renewcommand{\namefont}{\LARGE \bfseries}
\renewcommand{\titlefont}{\bf}
\renewcommand{\datesfont}{\bf}
\definecolor{linecolor}{RGB}{25,25,112}

% override defaults
\renewcommand{\employerfont}{}

% ugly kludge: we really have to define proper subsections in the .cls file
\renewcommand{\subsection}[1]{\section{\normalfont #1}}

\setlength{\parskip}{2ex}

\hypersetup{
  pdfauthor   = {Brit Butler},
  pdftitle    = {Curriculum Vitae: Brit Butler},
  pdfsubject  = {Curriculum Vitae},
  pdfkeywords = {Brit Butler, Curriculum Vitae, CV, resume},
}

\begin{document}

\name{Brit Butler}

\begin{resume}

% Specify the format of work entries
\begin{format}
\dates{l}\\
\title{l}\employer{r}\\
\body\\
\end{format}

%%%%%%%%%%%%%%%%%%%%%%%%%%%%%%%%%%%%%%%%%%%%%%%%%%%%%%%%%%%%%%%%%%%%%%%%%%%

%\begin{wrapfigure}{r}{40mm}
% \begin{center}
%  \includegraphics[width=0.1\textwidth]{lks}
% \end{center}
%\end{wrapfigure}

\section{Contact Information}

404-718-9378 \hfill {E-mail:} \href{mailto:brit@kingcons.io}{\nolinkurl{brit@kingcons.io}} \\
866 Woodland Ave SE \hfill {Code:} \url{https://github.com/kingcons/} \\
Atlanta, GA 30316 \hfill {Web:} \url{https://blog.kingcons.io/} \\

%%%%%%%%%%%%%%%%%%%%%%%%%%%%%%%%%%%%%%%%%%%%%%%%%%%%%%%%%%%%%%%%%%%%%%%%%%%

\section{\mbox{Work Experience}}

\title{Lead Instructor}
\employer{\llap{Flatiron School}}
\dates{October 2018 -- Present}
\begin{position}

\end{position}

\title{Senior Software Engineer}
\employer{\llap{Showcase IDX}}
\dates{June 2017 -- October 2018}
\begin{position}
  I took over Backend duties and DevOps for a Real Estate Search startup powering over 1000
  websites and helped scale the system to 1M+ page views a day. Major initiatives included moving
  the postgres instance from kubernetes to a dedicated Cloud SQL instance, writing a migration
  tool for customers on the legacy version of the product, improving the ingestion of data from
  MLS listing feeds, rebuilding the Elasticsearch cluster after a canary deploy went wrong,
  optimizing queries on our 30M+ record leads table, and other assorted production work.
  I did some ``Programmer Archaeology'' to relearn how to deploy the legacy system
  and switched it to using LetsEncrypt for SSL. Since I was the sole ops engineer, I thoroughly
  documented the architecture, its performance characteristics, and some gotchas before leaving.
\end{position}

\title{Lead Instructor}
\employer{\llap{The Iron Yard}}
\dates{December 2014 -- May 2017}
\begin{position}
  I taught immersive, full-time courses in Backend Engineering using Ruby/Rails and Frontend Engineering using Javascript/Angular.
  I was promoted to Lead Instructor after 6 months. As an instructor, I iterated
  on curriculum, lectured, graded assignments, and worked 1-on-1 with students
  during lab time. As a lead, I mentored a dozen new instructors at different
  campuses and served as a resource for classroom issues, struggling students,
  and other problems.
\end{position}

\title{Software Engineer}
\employer{\llap{Emcien}}
\dates{May 2013 -- October 2014}
\begin{position}
  I worked on several data analysis products written in Ruby/Rails and C.
  I added allocation tracking machinery to a modern 20k SLOC C project to aid
  in finding memory leaks and reducing the overall memory footprint. I took
  over maintenance of a legacy product, Mix, migrating from Ruby 1.8.7 to 1.9.3
  and overseeing numerous point releases. I also contributed substantial work
  to the primary product, Patterns, including the report download builders and
  storing report attributes in SQL shards.
\end{position}

\title{Software Engineer}
\employer{\llap{CMGdigital}}
\dates{May 2011 -- September 2012}
\begin{position}
  I worked on a 160k SLOC Python/Django project to serve Newspaper,
  TV, and Radio publishers. I had particular focuses on admin
  customization and data migration. I oversaw the migration of over
  100K users from multiple markets to our new CMS.
  I made broad improvements to our Brightcove video import
  scripts using celery and memcached. I also gave talks on
  \href{http://redlinernotes.com/docs/talks/opa.html}{``Programmer
    Archaeology''} and
  \href{http://redlinernotes.com/docs/talks/wosw.html}{the Economics
    of Open Source}.
\end{position}

%%%%%%%%%%%%%%%%%%%%%%%%%%%%%%%%%%%%%%%%%%%%%%%%%%%%%%%%%%%%%%%%%%%%%%%%%%%

\section{Education}

\title{B.Sc. in Computer Science}
\employer{\llap{Southern Polytechnic State University}}
\dates{January 2009 -- May 2011}
\begin{position}
  I transferred to SPSU in August 2007 to pursue Computer Science.
  After the first semester, I took a year off to work full-time and self-study.
  I returned in January 2009 and graduated in May 2011.
\end{position}

\title{Self-Study}
\dates{January 2008 -- August 2008}
\begin{position}
  After 3 years at Oglethorpe University studying literature
  I decided to self-study programming while working full-time.
  I did this throughout 2008 and have written about it, notably
  \href{https://blog.kingcons.io/posts/Leaving-College-to-Leverage-Compulsion.html}{here}
  and
  \href{https://blog.kingcons.io/posts/Spring-2008-Schedule-and-Syllabus.html}{here}.
  Example work from self-study can be found
  \href{https://blog.kingcons.io/posts/SICP-Section-13.html}{here}.
\end{position}

%%%%%%%%%%%%%%%%%%%%%%%%%%%%%%%%%%%%%%%%%%%%%%%%%%%%%%%%%%%%%%%%%%%%%%%%%%%

\pagebreak

% listings
\setlength{\parskip}{1ex}

%%%%%%%%%%%%%%%%%%%%%%%%%%%%%%%%%%%%%%%%%%%%%%%%%%%%%%%%%%%%%%%%%%%%%%%%%%%

\subsection{Passion Projects}

\title{Author}
\employer{\href{https://github.com/kingcons/study-group}{study-group}}
\dates{July 2018 -- Present}
\begin{position}
  I have organized and led an \href{https://sarabander.github.io/sicp/}{SICP} study group for
  myself and former coworkers and students. In addition to providing code feedback,
  I prepare the schedule, a meeting location, and host weekly discussion both online and in person.
\end{position}

\title{Author}
\employer{\href{https://github.com/kingcons/nescavation}{nescavation}}
\dates{August 2016 -- September 2016}
\begin{position}
  A NES simulator written during a semester break to practice a bigger project
  using ES6 classes and modules. I stalled out working on the graphics card
  when classes resumed. However, the light exposure to Canvas and requestAnimationFrame
  was valuable and I reused the decomposition of CPU opcode metadata from instruction
  definitions in a more recent hobby project called ``clones''.
\end{position}

\title{Author}
\employer{\href{https://github.com/kingcons/salty-runbooks}{salty-runbooks}}
\dates{November 2014 -- Present}
\begin{position}
  A set of Ansible playbooks to better familiarize myself with configuration management
  and help automate the setup and administration of my personal servers. It includes roles
  for retrieving SSL certificates with LetsEncrypt, setting up Postfix as a mail server,
  running an IRC bouncer, a blog generated with Lisp and served by nginx, and various
  web applications written in PHP, Python, and Node for streaming music and hosting media.
\end{position}

\title{Author}
\employer{\href{https://github.com/kingcons/coleslaw}{coleslaw}}
\dates{August 2012 -- November 2014}
\begin{position}
  coleslaw is static blogware a la Jekyll, written in Common Lisp. It
  supports publishing via git push, markdown with code highlighting extensions,
  extensible content types, theming, and various functionality through plugins
  all in under 1000 lines of code. It is also a good example of Object Oriented
  Programming in Lisp. Since ``you are what you document'' there is a thorough
  \href{https://github.com/kingcons/coleslaw/blob/master/docs/hacking.md}{Hacker's Guide to Coleslaw}
  explaining its internal design in addition to documentation of the Plugin API
  and extensions.
\end{position}

\title{Author}
\employer{\href{https://github.com/kingcons/cl-6502}{cl-6502}}
\dates{May 2011 -- May 2014}
\begin{position}
  cl-6502 is a MOS 6502 emulator, assembler, and disassembler written in
  Common Lisp. Inspired by Luke Gorrie's call for ``Readable Programs'' there
  is an \href{http://redlinernotes.com/docs/cl-6502.pdf}{annotated book} of the
  source code.  Some of the motivations behind cl-6502's creation are described
  \href{https://blog.kingcons.io/posts/Towards-Comprehensible-Computing.html}{here}.
  There is also a recorded talk on the project and related ideas called
  \href{https://vimeo.com/redline6561/on-programmer-archaeology}{``On Programmer Archaeology''}.
\end{position}

\title{Contributor}
\employer{Miscellaneous}
\dates{}
\begin{position}
  I have also contributed feature, portability, and documentation patches to:
  \begin{itemize}
    \item{Spacemacs, A community driven Emacs distribution}
    \item{Vacietis, a C to Common Lisp compiler}
    \item{st-json, a Common Lisp JSON library}
    \item{kardboard, a virtual Kanban web app made with Python and Flask}
    \item{pybrightcove, a Python interface to Brightcove's API}
  \end{itemize}
\end{position}

%%%%%%%%%%%%%%%%%%%%%%%%%%%%%%%%%%%%%%%%%%%%%%%%%%%%%%%%%%%%%%%%%%%%%%%%%%%

\section{Skills}

Programming Languages (intermediate): Common Lisp, Python, Ruby, Javascript \\
Programming Languages (novice): C, Scheme, Haskell, Factor, Elisp, Clojure \\
Markup Languages: HTML, CSS, SASS, LaTeX \\
Operating Systems: Mac OS X, Windows 3.11-7, Various Linux distributions
esp. Debian, Archlinux, Guix \\
Preferred Tools: Emacs, Git, Steel Bank Common Lisp, Debian \\

%%%%%%%%%%%%%%%%%%%%%%%%%%%%%%%%%%%%%%%%%%%%%%%%%%%%%%%%%%%%%%%%%%%%%%%%%%%

\section{Interests}

When I'm not in front of my computer, I like to make cocktails, spend
time with my partner and our goofy dogs, play Smash Brothers Melee,
and noodle on a modular synth. As far as Comp Sci topics, I am fascinated
by the implementation of dynamic, reflective languages such as Lisp and Smalltalk
and the careful interplay between compiler and runtime to make them fast.

%%%%%%%%%%%%%%%%%%%%%%%%%%%%%%%%%%%%%%%%%%%%%%%%%%%%%%%%%%%%%%%%%%%%%%%%%%%

\section{References}

Available on request.

\end{resume}
\end{document}
